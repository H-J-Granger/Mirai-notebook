% !TEX root = ../main.tex
\chapter*{前言}

这篇笔记是笔者在学习《数学分析》课程中的笔记。笔记内容自中国科学技术大学(下称科大)的《数学分析(B1)》(课程号 MATH1006)、《数学分析(B2)》(MATH1007)、《数学分析(B3)》(MATH1008)的课程设置出发,根据各参考书增添了许多内容。
  
\index{Bourbaki 危险区符号}
% 这篇笔记的特色在于在传统的数学分析课程之外,提供许多注记。
为辅助理解,行文过程中的许多例子可能会超出当前的可证明范围,这些例子会使用“Bourbaki\footnotemark{} 危险区符号”(Bourbaki Dangerous Bend)\footnotemark{}来标注,如下例所示:

\footnotetextpre[1]{Nicolas~Bourbaki (1934--),Bourbaki 学派 (Bourbaki group) 的共用笔名。}
\footnotetext{这里用的符号实际上是 Knuth\footnotemark{} 在 \cite{knuth86} 中使用的。}
\footnotetext{Donald~E.~Knuth (1938--),美国计算机科学家、数学家。}


\index{Fermat 大定理}
\begin{dtheorem*}[Fermat\footnotemark{} 大定理,Wiles\footnotemark{}]
  对任意大于 \(2\) 的整数,都不存在正整数 \(x,y,z\) 满足 \(x^n+y^n=z^n\)。
\end{dtheorem*}

\footnotetextpre[1]{Pierre de Fermat (1607--1665),法国数学家。}
\footnotetext{Sir Andrew John Wiles (1953--),英国数论学家。}

为了保证不存在循环论证,前面的证明也不会使用后面的结论。使用后面结论的例子也会被打上标记。行文需要的代数、数论等其他知识均在附录中有所给出。

% TODO: 完成与线性代数之间逻辑关系的图示,参见 Hungerford, Algebra page xvi.
% TODO: 完成对大致内容的介绍。

考虑到篇幅、时间、精力所限,一些证明的细节留予读者验证。读者也通常能够在公开的参考资料中找到更详细的证明。

最后,我要感谢在科大教授《数学分析(B1)》的吴健老师、陈奕杰助教、杨笛龙助教、孙孝同助教;教授《数学分析(B2)》的张明波老师、曾华锋助教、周汉鹏助教;教授《数学分析(B3)》的许斌老师、左一泓助教、吴东润助教,以及教授《数学强基讨论班 I》《数学强基讨论班 II》的李沐西老师。\footnotemark{}他们的课程让我受益匪浅,很多例子也来自于他们的讲义、课件和习题。

\footnotetext{这些课程的课程号分别为 MATH1006.15.2023FA、MATH1007.06.2024SP、MATH1008.02.2024FA、MATH2802.01.2023FA 和 MATH2802.01.2024SP。}

\begin{flushright}
  栗山~未来\footnotemark

  2024 年 11 月,于合肥
\end{flushright}
\footnotetext{栗山~未来,中国科学技术大学少年班学院,\url{mailto:kuriyama.mirai@mail.ustc.edu.cn}。}