% !TEX root = ../main.tex
\chapter{公理化集合论}

\newcommand{\vdashLA}{\vdash_{\text{LA}}}

本章主要目标是从基础的逻辑出发,给出带有选择公理的 Zermelo\footnotemark{}--Fraenkel\footnotemark{} 集合论体系。\footnotemark{}

务必注意一点:本章在完整给出 ZFC 以前的讨论并不太严格。对 ZFC 严格的讨论大大超出了本书的范围,本章的讨论可以视为一种对 ZFC 的更加深刻的理解,以帮助读者更好地理解 ZFC。

\footnotetextpre[2]{Ernst Friedrich Ferdinand Zermelo (1871--1953),德国逻辑学家、数学家。}
\footnotetextpre[1]{Abraham Fraenkel (1891--1965),以色列数学家。}
\footnotetext{计算机科学界有说法“程序员最大的浪漫是在自己写的 CPU 上运行自己写的操作系统,再在其上运行自己写的编译器所编译出的程序”。从形式逻辑出发推演出数学分析所有常用的结论是否也算一种浪漫呢?}

% !TEX root = ../main.tex
\section{语言与逻辑}

王新茂老师有名言:“学数学就是学说话”。本节我们来考察数学这门语言的最基础的部分——逻辑。

本节主要参考自 \cite[chp. 1]{takeuti12}。  

逻辑作为一门语言,自然有其最基本的词语(符号)。

\index{bianliang@变量!ziyou@自由\~{}}
\index{bianliang@变量!yueshu@约束\~{}}
\index{fuhao@符号!weici@谓词\~{}}
\index{fuhao@符号!luoji@逻辑\~{}}
\index{fuhao@符号!fuzhu@辅助\~{}}
\begin{definition}[逻辑的基本符号]\emptyline
  本章我们表记 \textbf{自由变量}(free variable) 为 \(a_0,a_1,\cdots\);我们表记 \textbf{约束变量}(bound variable) 为 \(x_0,x_1,\cdots\)。我们使用一个 \textbf{谓词符号}(predicate symbol) \(\in\)。我们还有一些 \textbf{逻辑符号}(logical symbol) \(\lnot,\land,\lor,\rightarrow,\leftrightarrow,\forall,\exists\) 和一些 \textbf{辅助符号}(auxiliary symbol) \((\), \()\), \([\), \(]\)。
\end{definition}

\index{bianliang@变量!yuan@元\~{}}
\begin{remark}[元变量]
  我们会用 \textbf{元变量}(metavariable) \(a,b,c,d\) 来表示某些自由变量 \(a_0,a_1,\cdots\);用元变量 \(x,y,z\) 来表示某些约束变量 \(x_0,x_1,\cdots\);用 \(\varphi,\psi,\eta\) 来表示某些良构公式(定义见下)。
\end{remark}

\index{lianggougongshi@良构公式}
\index{lianggougongshi@良构公式!yuanzide@原子的\~{}}
\begin{definition}[良构公式] \label{def:良构公式}
  我们定义 \textbf{良构公式}(well-formed formula,简称公式) 如下:
  \begin{enumerate}
    \item 若 \(a,b\) 是自由变量,则 \([a\in b]\) 是一个良构公式。这样的公式被称为是 \textbf{原子的}(atomic)。
    \item 若 \(\varphi,\psi\) 是良构公式,则 \(\lnot\varphi,[\varphi\lor\psi],[\varphi\land\psi],[\varphi\rightarrow\psi],[\varphi\leftrightarrow\psi]\) 都是良构公式。
    \item 若 \(\varphi\) 是公式,则 \((\forall x)\varphi(x)\) 和 \((\exists x)\varphi(x)\) 都是公式,其中 \(\varphi(x)\) 表示用 \(\varphi\) 中某些自由变量 \(a\) 用 \(x\) 替换所得到的字符串。
    \item 任意不能用 (a)--(c) 导出的字符串都不是良构公式。
  \end{enumerate}
\end{definition}

\begin{remark}
  上述定义中的符号较为繁琐。在不引起歧义的前提下,我们也会省略一些辅助记号。譬如,我们更倾向于写 \(a_0\in a_1\) 而不是 \([a_0\in a_1]\);写 \(a_0\in a_0\rightarrow a_0\in a_1\) 而不是 \([[a_0\in a_0]\in [a_0\in a_1]]\)。
\end{remark}

接下来我们来给出逻辑的公理。

\index{gongli@公理!luoji@逻辑\~{}}
\begin{axiom}[逻辑公理, logical axiom]\emptyline \label{axiom:逻辑公理}
  \begin{enumerate}
    \item \(\varphi\to[\psi\to\varphi].\)(真命题被一切命题蕴含)
    \item \(\Bigl[\varphi\to[\psi\to\eta]\Bigr]\to\Bigl[[\varphi\to\psi]\to[\varphi\to\eta]\Bigr].\) (命题真性的传递性)
    \item \([\lnot \varphi\to\lnot \psi]\to[\psi\to\varphi].\)(命题等价于其逆否)
    \item \((\forall x)[\varphi\to\psi(x)]\to[\varphi\to(\forall x)\psi(x)]\),其中 \(\psi(x)\) 中替换掉的 \(\psi\) 中的所有自由变量 \(a\) 都不出现在 \(\varphi\) 中。(蕴含与任意的可交换性)
    \item \((\forall x)\varphi(x)\to\varphi(a)\),其中 \(a\) 可以取任意一个自由变量;\(\varphi(a)\) 表示将 \(\varphi(x)\) 中所有的 \(x\) 替换为 \(a\)。
  \end{enumerate}
\end{axiom}

接下来我们介绍“逻辑推理”的全部规则。

\index{tuiliguize@推理规则}
\begin{axiom}[推理规则]\emptyline 
  \begin{enumerate}
    \item 我们从 \(\varphi\) 和 \(\varphi\to\psi\) 中推出 \(\psi\)。
    \item 我们从 \(\varphi\) 中推出 \((\forall x)\varphi(x)\)。
  \end{enumerate}
\end{axiom}

\index{dingli@定理}
\index{tuidaofuhao@推导符号}
\index{yuanyuju@元语句}
\begin{definition}[定理]
  我们称一个公式 \(\varphi\) 是一个 \textbf{定理}(theorem),当且仅当其可以通过推理规则从逻辑公理中推出。我们用 \textbf{推导符号}(turnstile) \(\vdash\) 来标记一个公式是一个定理:具体而言,我们用 \textbf{元语句}(metastatement) \(\vdash\varphi\) 来表示 \(\varphi\) 是一个可以从逻辑公理和后文一些不是逻辑公理的公理推导出的公式。我们用 \(\vdashLA \varphi\) 来表示 \(\varphi\) 可以仅用逻辑公理推导出。我们用称两个公式 \(\varphi\) 和 \(\psi\) 是 \textbf{逻辑等价的}(logically equivalent),当且仅当 \(\vdashLA \varphi\leftrightarrow\psi\)。
\end{definition}
% !TEX root = ../main.tex
\section{相等}

接下来,我们来引入一个很重要的概念:相等。本节主要参考 \cite[chp. 2]{takeuti12}。

\index{xiangdeng@相等}
\begin{definition}[相等]
  称 \(a\) 和 \(b\) \textbf{相等}(equality),即
  \begin{equation*}
    (\forall x)[x\in a\leftrightarrow x\in b].
  \end{equation*}
\end{definition}

\begin{proposition}[相等的性质]\emptyline \label{prop:相等的性质}
  \begin{enumerate}
    \item \(a=a\).
    \item \(a=b\to b=a\).
    \item \(a=b\land b=c\to a=c\).
  \end{enumerate}
\end{proposition}

\begin{proof}\emptyline
  \begin{enumerate}
    \item \((\forall x)[x\in a\leftrightarrow x\in a]\).
    \item \((\forall x)[x\in a\leftrightarrow x\in b]\rightarrow (\forall x)[x\in b\leftrightarrow x\in a]\).
    \item \((\forall x)[x\in a\leftrightarrow x\in b]\land(\forall x)[x\in b\leftrightarrow x\in c]\to (\forall x)[x\in a\leftrightarrow x\in c]\).\qedhere
  \end{enumerate}
\end{proof}

我们肯定会期待我们定义的“相等”满足如下性质:
\begin{equation*}
  a=b\rightarrow [\varphi(a)\leftrightarrow \varphi(b)].
\end{equation*}

接下来我们要给出一个更弱的命题作为公理,将上式作为其的一个推论。

\begin{axiom}[外延公理] 相等的元素由同样的元素组成。 \label{axiom:外延公理}
  \begin{equation*}
    a=b\land a\in c\rightarrow b\in c.
  \end{equation*}
\end{axiom}

\begin{lemma}
  \(a=b\to[a\in c\leftrightarrow b\in c]\).
\end{lemma}
\begin{proof}
  从公理 \ref{axiom:外延公理} 和命题 \ref{prop:相等的性质}b 立即得出。
\end{proof}

\begin{theorem}
  \(a=b\to[\varphi(a)\leftrightarrow \varphi(b)]\).
\end{theorem}
\begin{proof}
  我们对 \(\varphi\) 所拥有的逻辑符号数量 \(n\) 作归纳。当 \(n=0\) 时,\(\varphi\) 只能形如 \(c\in d,c\in a,a\in c,a\in a\) 之一。显然 \(a=b\to[c\in d\leftrightarrow c\in d]\);从相等的定义知 \(a=b\to[c\in a\leftrightarrow c\in b]\);由引理知 \(a=b\to[a\in c\leftrightarrow b\in c]\)。从等式的定义以及引理我们分别知道
  \begin{equation*}
    a=b\to[a\in a\leftrightarrow a\in b],\qquad 
    a=b\to[a\in b\leftrightarrow b\in b],
  \end{equation*}
  于是有 \(a=b\to[a\in a\leftrightarrow b\in b]\)。

  对于 \(n>0\) 的情况,我们可以利用
  \begin{equation*}
    \vdashLA \psi\lor\eta\leftrightarrow \lnot[\lnot\psi\land\lnot\eta],\qquad
    \vdashLA (\exists x)\psi(a,x)\leftrightarrow \lnot(\forall x)[\lnot\psi(a,x)]
  \end{equation*}
  将其变为逻辑符号仅含 \(\lnot,\lor,\forall\) 的公式。于是,根据归纳假设,拥有 \(n\) 个逻辑符号的 \(\varphi(a)\) 总形如
  \begin{equation*}
    \text{(a) }\lnot\psi(a),\qquad
    \text{(b) }\psi(a)\land\eta(a),\qquad 
    \text{(c) }(\forall x)\psi(a,x)
  \end{equation*}
  之一。

  对于 (a),(b) 两种情况,由于 \(\psi(a),\eta(a)\) 的逻辑符号个数小于 \(n\),故根据归纳假设我们有
  \begin{equation*}
    a=b\rightarrow [\psi(a)\leftrightarrow\psi(b)],\qquad
    a=b\rightarrow [\eta(a)\leftrightarrow\eta(b)],
  \end{equation*}
  于是根据 \(\lnot,\land\) 的性质有
  \begin{equation*}
    a=b\rightarrow [\lnot\psi(a)\leftrightarrow\lnot\psi(b)],\qquad
    a=b\rightarrow [\psi(a)\land\eta(a)\leftrightarrow\psi(b)\land\eta(b)].
  \end{equation*}

  接下来考虑 (c) 的情况。若 \(\varphi(a)\) 形如 \((\forall x)\psi(a,x)\),则我们取一个自由变量 \(c\) 既不在 \(\psi(a,x)\) 内出现过,也不为 \(a,b\)。那么,根据归纳假设,我们有
  \begin{equation*}
    a=b\rightarrow [\psi(a,c)\leftrightarrow\psi(b,c)].
  \end{equation*}

  接下来运用逻辑公理 \ref{axiom:逻辑公理}d,有
  \begin{equation*}
    (\forall x)[\psi(a,x)\leftrightarrow \psi(b,x)]\to
    [(\forall x)\psi(a,x)\leftrightarrow (\forall x)\psi(b,x)],
  \end{equation*}
  于是我们有
  \begin{equation*}
    a=b\to[(\forall x)\psi(a,x)\leftrightarrow (\forall x)\psi(b,x)].\qedhere
  \end{equation*}
\end{proof}

\begin{remark}
  外延性保证一个集合仅由其组成元素确定。如果将 \(=\) 引入基本逻辑符号,有些读者可能会认为外延公理可以被舍弃。然而,Scott 在 \cite[pp. 115--131]{scott62} 中证明,舍弃外延公理必然会导致弱化整个逻辑系统。于是,就算 \(=\) 被引入进基本逻辑符号,外延公理仍然是必要的,参见 \cite{quine69}。
\end{remark}