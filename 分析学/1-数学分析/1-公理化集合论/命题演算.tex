% !TEX root = ../main.tex
\section{命题演算}

王新茂老师有云:“学数学就是学说话”。数学的证明,通常使用的是可以化为逻辑语言的自然语言。接下来我们开始考察一种最基础的逻辑语言:命题演算。

本节主要参考自 \cite[2.1]{wang18}。

我们可以视命题演算为一种形式语言。

\index{mingtiyansuanxingshixitong@命题演算形式系统}
\index{mingtibianyuan@命题变元}
\index{mingtilianjieci@命题连接词}
\begin{definition}[命题演算形式系统] \label{def:命题演算形式系统}
  \textbf{命题演算形式系统} 是一种语言,有如下字母表:
  \begin{enumerate}
    \item \(x,y,z,x_1,y_1,z_1,\cds\) 作为 \textbf{命题变元};
    \item \(\lnot,\land,\lor,\Rightarrow,\Leftrightarrow\) 作为 \textbf{命题连接词};
    \item 辅助记号 \((\) 和 \()\) 用作确定符号结合的顺序,
  \end{enumerate}
  同时,我们使用如下的规则构成公式:
  \begin{enumerate}
    \item 每个命题变元都是一个公式,称为 \textbf{简单公式};
    \item 若 \(p,q\) 是公式,则 \(\lnot p,(p\land q),(p\lor q),(p\Rightarrow q),(p\Leftrightarrow q)\) 都是公式。
  \end{enumerate}
\end{definition}