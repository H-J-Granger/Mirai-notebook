% 这里最好放 \begin{document} 以前的内容。

\usepackage{amsmath,amssymb,amsthm}
\usepackage{titlesec,booktabs,tikz}
\usepackage[a4paper,margin=1in]{geometry}

\usepackage{chngcntr}
\numberwithin{equation}{section}

\usepackage[acronym]{glossaries}

\usepackage[makeindex]{imakeidx}  % 索引表
% myindex_style.ist 是用作给索引提供字母索引的。
% 下面是一些例子。
% 有个 bug:怀疑第一次使用字母索引需要一个纯 ASCII 单词(如 \index{PDE})来启动。

\iftrue
% 汉字开头的内容要手动提供拼音来索引。
% 尽管 texindy 理论上支持 -C chinese -L pinyin 来自动生成拼音索引,但是实际上无用。
% 汉字索引统一使用拼音。和制汉字优先使用新华社提供的拼音。

% \index{lishan@栗山}  % @前面的是用作索引的字符
% \index{lishan@栗山!weilai@未来}  % !是用作缩进。
% \index{lishan@栗山!meixing@美幸}
% \index{shenyuanqiuren@神原秋人} 
% \index{woheni@我和你}



% 理论上应该写 
%     \index{Lagrangezhongzhidingli@Lagrange 中值定理}
% 之类的,但是开头的 Lagrange 已经足够索引了。
  \index{Lagrange 中值定理}  
% 行内公式几乎必须要用手动索引。
  \index{sigmadaishu@\(\sigma\)-代数} 
% 注音字母是不需要索引的(不过场景很少)
  \index{\'Sniatycki 定理}
\fi  

\bibliographystyle{plain}


% 不能放在前面,不然 Index 就没有链接了。
\usepackage[hidelinks]{hyperref}

%# TODO: 符号表(可以用 glossary)

\usepackage{mirai}